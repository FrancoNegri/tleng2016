\subsection{Descripción General}
\subsubsection{Gramática inicial}
Comenzamos el armado de la gramática definiendo los tipos básicos que debíamos aceptar en nuestro lenguaje.
\\
Así definimos los tipos Bool, String, Float, Bool, Int, Vector, Registro. De allí el siguiente paso fue ver que operaciones se le podía realizar a cada uno de ellos.
\\
Mientras realizábamos esto empezamos a notar que necesitaríamos determinar la precedencia de estas operaciones y la asociatividad. Para esto nos basamos en la precedencia de c++ \footnote{\url{http://en.cppreference.com/w/cpp/language/operator_precedence}} dado lo parecido que es al lenguaje de entrada.
\\

Una vez que todas las expresiones del lenguaje estaban completas, lo pasamos a código Python usando la herramienta $PLY$. Ya terminado el código, nos encontramos con que la gramática tenía muchos mas conflictos que los que se podía ver a simple vista y por lo tanto no era LALR. 

\subsubsection{Gramática ambigua}

Luego de resolver algunos conflictos, nos dimos cuenta de que era un proceso complicado pues cada conflicto implicaba añadir mas reglas a la gramática haciendo que sea mas compleja y difícil de entender.

Después de leer la documentación de ply\footnote{ \url{http://www.dabeaz.com/ply/ply.html}} y observar que ply puede encargarse de los conflictos de una gramática ambigua usando reglas de asociatividad y precedencia, decidimos hacer una gramática ambigua sencilla y aprovechar esta característica de ply.

\subsubsection{Gramática no ambigua}

Si bien esto solucionó todos los conflictos de una manera sencilla, por desgracia, nos comunicaron que lo que se pedía era una gramática \textbf{no} ambigua. Por lo tanto decidimos tirar la gramática y comenzar de nuevo.

Con el tiempo contado y sin ideas mejores nos dimos cuenta de que la gramática inicial no estaba tan mal, tenía toda la precedencia definida de las expresiones y que una solución rápida era hacer la gramática menos restrictiva, es decir, que acepte mas cadenas de lo que se pida y rechazar las que no pertenecen al lenguaje mediante atributos. La tarea de los atributos (además de armar el código indentado) es chequear los tipos de las expresiones para que, si una operación acepta unos tipos determinados y se le pasa una expresión de otro tipo, lance una excepción de error de tipos. Dicha gramática se encuentra definida en la sección \textbf{5}

\subsection{Descripción de la gramática implementada:}

\subsubsection{Valores}
Los valores son cualquier expresión que denote un valor (por ejemplo 3 * 4 + 3). La primer idea era definir valores como \textbf{cualquier} expresión (para poder usarla como valor en asignaciones, en funciones como parámetro, etc.):

$Valores \rightarrow ExpresionMatematica \  | \  ExpresionBooleana \  | \ ExpresionString  \ | \ ExpresionVector \ | \ \\ ExpresionRegistro \ | \ OperacionVariables $

Donde cada producción que empieza con $Expresion$ denota una expresión de un tipo determinado que también puede generar los valores primitivos (bool, string, float, bool, int, vec, reg), variables (id) y además funciones que retornen valores a su tipo correspondiente (por ejemplo $capitalizar$ en $Expresionstring$). $OperacionVariables$ genera las expresiones que usan operaciones de variables (asignación, incremento, decremento, etc). 

El problema con esto es que \textbf{id} se puede generar en todas las expresiones salvo en la última. Si se tiene una producción $A \rightarrow Valores$, entonces se puede llegar de 5 maneras distintas a \textbf{id} (es decir, hay mas de una posible derivación mas a la izquierda para la cadena \textit{id}) y, por lo tanto, hay conflictos (la gramática es ambigua). 

Por lo tanto decidimos sacar a \textbf{id} de cada producción. El problema es que se tiene que seguir produciendo expresiones complejas con \textbf{id} (por ejemplo $id$ $and$ $id$).

Para solucionar esto, decidimos sacar todos los valores primitivos de cada expresión y colocarlo en $Valores$. Además hay que observar que una función de un determinado tipo puede devolver una variable de ese tipo, por lo que decidimos sacar también a esas funciones. Lo mismo pasa con los operadores ternarios. De esta forma, cada producción $Expresion$ genera expresiones con al menos un operador de su tipo:

$Valores \rightarrow ExpresionMatematica \ | \ ExpresionBooleana \  | \ ExpresionString \  | \ ExpresionVector \ | \  $

\hspace{15mm} $| ExpresionRegistro  \ | \ OperacionVariables \ | \ bool \ | \ string \ | \ float \ | \ bool \ | \ int \ | \ vec \ | \ reg  $

\hspace{15mm} $| \ id \ | \ FuncReturn \ | \ Ternario $

$FuncReturn \rightarrow FuncInt \ | \ FuncString \ | \ FuncBool \ | \ FuncVector $

$Ternario \rightarrow  \ TernarioMat \ | \ TernarioBool \ | \ TernarioString \ | \ TernarioVector \ | \ TernarioRegistro $


\subsubsection{Expresiones Matemáticas}
Para las expresiones matemáticas consideramos la siguiente tabla de precedencia:

\begin{table}[htbp]
\begin{center}
\begin{tabular}{|l|l|l|}
\hline
Tipo & Operador & Asociatividad \\
\hline \hline
Binario & +,- & izquierda \\ \hline
Binario & *,/,\% & izquierda \\ \hline
Binario & \^{} & izquierda \\ \hline
Unario & +,- &  \\ \hline
Unario & () &  \\ \hline

\end{tabular}
\caption{Tabla de menor a mayor precedencia}
\label{Precedencia}
\end{center}
\end{table}

La producción que se encarga de realizar estas expresiones es eMat. 

La primera aproximación fue hacer la gramática para poder respetar la precedencia y asociatividad de los operadores. 

Este proceso se logra teniendo en cuenta que se puede fijar la precedencia (por ejemplo entre $+$ y $*$), con las reglas de la gramática. Como bien se sabe, el operador $+$ tiene menos precedencia que el $*$. Una gramática que respeta esto es la siguiente:

\begin{center}
$A \rightarrow A + B$ \\
$B \rightarrow B * C$ \\
$C \rightarrow int | float$ 
\end{center}
Esto se debe a que antes de calcular la suma entre $A$ y $B$, hay que parsear la producción B (pues es un parser ascendente). 

Esto también respeta la asociatividad de ambos operados (que es a izquierda) por la misma razón que antes: La única forma de colocar mas de un operador $+$ es usando la producción $A$ y, por lo tanto, se tiene que parsear $A$ antes que calcular la suma.

Este proceso se realiza con todos los operadores, de manera que si 2 operadores tienen la misma precedencia, entonces parten de un mismo no terminal.

A esta gramática le falta la posibilidad de agregar paréntesis para poder cambiar la precedencia. Una forma de hacerlos es agregarlos a $C$ para que estos tengan mas prioridad que todas las operaciones.

Además también falta poder agregar variables. Para esto agregamos un \textbf{no-terminal} a $C$ llamado id que las representa. La gramática resultante es esta:
\begin{center}
$A \rightarrow A + B$ \\
$B \rightarrow B * C$ \\
$C \rightarrow (A) | int | float | id$ 
\end{center}

Para evitar problemas de ambigüedad, decidimos hacer que eMat devuelva expresiones con al menos un operador matemático. Esto es por que en la producción valores se tienen las variables (correspondientes al token ID) y estas
también se generarían en las expresiones de los otros tipos. Ver explicación en \textbf{4.1}

\subsubsection{Expresiones booleanas}
Las expresiones booleanas también se soluciona de la misma manera que las expresiones matemáticas. Para mas información ver sección 5.

\subsubsection{Expresiones de string}
En principio consideramos la siguiente gramática:

$ExpresionString \rightarrow ExpresionString + ExpresionString \ | \ string$ \\

Esta gramática nos planteó un problema en el parser debido a que comparte un operador con las expresiones matemáticas. El problema viene cuando se tiene una expresión $a+b$ en donde no está claro que tipo le corresponde e introduce una ambigüedad.

Por lo tanto decidimos fusionar a las expresiones de string con las expresiones matemáticas y restringir (mediante chequeo de tipos) que un string no se le puede hacer otra operación que no sea la suma.

\subsubsection{Problema del ``Dangling else'':}

Otro problema que encontramos ya avanzados en el proceso de desarrollo de la gramática fue el problema del ``Dangling else'' \footnote{\url{https://en.wikipedia.org/wiki/Dangling_else}} que consiste en que los else opcionales en un if hacen que la gramática sea ambigua. 
\\
Por ejemplo: 

if cond1 then if cond2 then sentencia1 else sentencia2 \\

La ambigüedad surge de no poder decidir si la sentencia2 se ejecuta como rama alternativo a cond1( es decir, cuando esta condición es falsa), o bien como la rama alternativa del \textit{if} asociado a \textbf{cond2}. Esto último es posible ya que el \textit{if} de \textbf{cond1} puede prescindir del bloque \textit{else}.
Para solucionar esto, construimos una solución basada en el concepto de lineas abiertas o cerradas. La primera, son todas aquellas que contienen al menos un \textit{if} que no tiene un \textit{else} correspondiente o bien un bucle con linea abierta. La segunda mencionada, son aquellas que resultan de tener sentencias simples, comentarios, o bien: \\

if (cond1) bloqueCerrado else bloqueCerrado \\

Donde bloqueCerrado puede ser cualquier conjunto de sentencias escritas entre llaves, sentencias simples o , razonando inductivamente, lineas cerradas. También se incluyen los bucles con lineas cerrados. Finalmente, la situación se complejizó un poco ya que hay que considerar las distintas combinaciones recién mencionadas, sumando la posibilidad de tener, o no, comentarios en cada uno de estos bloques.


\subsection{Gramática Final}
A continuación se define la gramática utilizada para construir el parser. La misma es \textbf{no ambigua} y \textbf{LALR}. La garantía de esto es que ply acepta gramáticas de este tipo, y nuestra implementación no arroja conflictos \textbf{shift/reduce} o \textbf{reduce/reduce} \\

\linespread{1.125}\selectfont

$G$|$<V_{t}$, $V_{nt}$, $g$, $P>$

$V_{t}$ es el conjunto de símbolos terminales dado por los símbolos en \textbf{mayúsculas} que aparecen 

en las producciones.


$V_{nt}$ es el conjunto de símbolos no-terminales dado los literales(operadores) y los símbolos

en \textbf{minúsculas} que aparecen en las producciones.



$P$ es el conjunto de producciones dadas a continuación$|$\\


\textbf{Sentencias y estructura general} \\

g $\rightarrow$ linea g $|$ COMMENT g $|$ empty 

linea $\rightarrow$ lAbierta $|$ lCerrada \\


\textit{Linea Abierta: Hay por lo menos un IF que no matchea con un else} 

lAbierta $\rightarrow$ IF (cosaBooleana) linea $|$

\hspace{15mm}$|$ IF (cosasBooleana) $\{$ g $\}$ ELSE lAbierta

\hspace{15mm}$|$ IF (cosasBooleana) $\{$ g $\}$ ELSE lAbierta

\hspace{15mm}$|$ IF (cosasBooleana) $\{$ g $\}$ 

\hspace{15mm}$|$ loop Labierta \\

\textit{Las siguientes son las variantes de tener bloques cerrados else bloques cerrados.}

\textit{Un bloque cerrado puede ser una sentencia única o un bloque entre llaves.}

\textit{En cada uno de estos casos puede haber, o no, comentarios. De ahí todas estas combinaciones.} 


lCerrada $\rightarrow$ sentencia

\hspace{15mm}$|$ IF (cosaBooleana) $\{$ g $\}$ ELSE $\{$ g $\}$

\hspace{15mm}$|$ IF ( cosaBooleana ) lCerrada ELSE $\{$ g $\}$ 

\hspace{15mm}$|$ IF ( cosaBooleana ) COMMENT com lCerrada ELSE $\{$ g $\}$ 

\hspace{15mm}$|$ IF ( cosaBooleana ) $\{$ g $\}$ ELSE lCerrada 

\hspace{15mm}$|$ IF ( cosaBooleana ) lCerrada ELSE lCerrada 

\hspace{15mm}$|$ IF ( cosaBooleana ) COMMENT com lCerrada ELSE lCerrada 

\hspace{15mm}$|$ IF ( cosaBooleana ) lCerrada ELSE  COMMENT com lCerrada 

\hspace{15mm}$|$ IF ( cosaBooleana ) COMMENT com lCerrada ELSE  COMMENT com lCerrada

\hspace{15mm}$|$ loop $\{$ g $\}$ 

\hspace{15mm}$|$ loop lCerrada 

\hspace{15mm}$|$ loop COMMENT com lCerrada

\hspace{15mm}$|$ DO $\{$ g $\}$ WHILE ( valores ) ;
  
\hspace{15mm}$|$ DO lCerrada WHILE ( valores ) ;  
  
\hspace{15mm}$|$  DO COMMENT com lCerrada WHILE ( valores ) ;

com $\rightarrow$  COMMENT com $|$ $\lambda$ \\

\textbf{Sentencias básicas:}

sentencia $\rightarrow$ varsOps $|$ func ; $|$ varAsig ; $|$ RETURN; $|$ ; \\


\textbf{Bucles:}

loop $\rightarrow$ WHILE (valores) $|$ FOR (primerParam ; valores ; tercerParam)

primerParam $\rightarrow$  varAsig $|$ $\lambda$

tercarParam $\rightarrow$  varsOps $|$ varAsig $|$ func $|$ $\lambda$

cosaBooleana $\rightarrow$ expBool $|$ valoresBool \\
  

\textbf{Funciones:}

func $\rightarrow$ FuncReturn $|$ FuncVoid

funcReturn $\rightarrow$ FuncInt $|$ FuncString $|$ FuncBool

funcInt $\rightarrow$ MULTIESCALAR( valores, valores param)

funcInt $\rightarrow$ LENGTH( valores)

funcString $\rightarrow$ CAPITALIZAR(valores)

funcBool $\rightarrow$ colineales(valores,valores )

funcVoid $\rightarrow$ print(Valores) 

param $\rightarrow$ valores $|$ $\lambda$ \\

\textbf{Vectores y variables:}

vec $\rightarrow$ [elem]

elem $\rightarrow$ valores,elem $|$ valores

vecval $\rightarrow$ id [expresion] $|$ vec [expresion] $|$ vecVal [expresion] $|$ ID[INT]

expresion $\rightarrow$ eMat
$|$  expBool
$|$  funcReturn
$|$  reg
$|$  FLOAT $|$
$|$STRING $|$
$|$  RES

\hspace{15mm}$|$   BOOL $|$  varYVals $|$  varsOps $|$  vec $|$  atributos $|$  ternario 
 
valores $\rightarrow$  varYVals
$|$  varsOps
$|$  eMat
$|$  expBool
$|$  funcReturn
$|$  reg
$|$  INT
$|$  FLOAT
  
  \hspace{15mm}$|$  STRING
  $|$  BOOL
  $|$  ternario
  $|$  atributos
  $|$  vec
  $|$  RES
 
atributos $\rightarrow$ ID.valoresCampos $|$ reg.valoresCampos

valoresCampos $\rightarrow$ varYVals $|$ END $|$ BEGIN \\

\textbf{Operadores ternarios:}

ternario $\rightarrow$ ternarioMat $|$ ternarioBool $|$ (ternarioBool) $|$ (ternarioMat) 

\hspace{15mm}$|$ ternarioVars  $|$ (ternarioVars)

ternarioVars $\rightarrow$ valoresBool ? valoresTernarioVars : valoresTernarioVars  

  \hspace{15mm}$|$  valoresBool ? valoresTernarioVars : valoresTernarioMat 
  
  \hspace{15mm}$|$  valoresBool ? valoresTernarioMat : valoresTernarioVars
  
  \hspace{15mm}$|$  valoresBool ? valoresTernarioVars : valoresTernarioBool 
  
  \hspace{15mm}$|$  valoresBool ? valoresTernarioBool : valoresTernarioVars
  
  \hspace{15mm}$|$  expBool ? valoresTernarioVars : valoresTernarioVars
  
  \hspace{15mm}$|$  expBool ? valoresTernarioVars : valoresTernarioMat 
  
  \hspace{15mm}$|$  expBool ? valoresTernarioMat : valoresTernarioVars
  
  \hspace{15mm}$|$  expBool ? valoresTernarioVars : valoresTernarioBool 
  
  \hspace{15mm}$|$  expBool ? valoresTernarioBool : valoresTernarioVars  
  

  
valoresTernarioVars $\rightarrow$ reg $|$ vec $|$ ternarioVars $|$ (ternarioVars) $|$ atributos 

\hspace{15mm}$|$ varsOps  $|$ varYVals $|$ RES

TernarioMat $\rightarrow$ valoresBool ? valoresTernarioMat : valoresTernarioMat  

\hspace{15mm}$|$ expBool ? valoresTernarioMat : valoresTernarioMat
  
valoresTernarioMat $\rightarrow$ INT $|$ FLOAT $|$ funcInt $|$ STRING $|$ eMAt 

\hspace{15mm}$|$ ternarioMat $|$ (ternarioMat)

ternarioBool $\rightarrow$ valoresBool ? valoresTernarioBool : valoresTernarioBool  

\hspace{15mm}$|$ expBool ? valoresTernarioBool : valoresTernarioBool
  
valoresTernarioBool $\rightarrow$ BOOL $|$ funcBool $|$ ternarioBool $|$ ( ternarioBool )$|$ expBool \\
   

\textbf{varYVals:}

varYVals $\rightarrow$ ID $|$ vecVal $|$ vecVal.varYVals \\


\textbf{Registros:} 

reg $\rightarrow$ $\{$ campos $\}$

campos $\rightarrow$ ID:valores, campos $|$ ID:valores \\


\textbf{Operadores de variables:}

varsOps$\rightarrow$ MENOSMENOS varYVals $|$ MASMAS varYVals

\hspace{15mm}$|$ varYVals MASMAS $|$ varYVals MENOSMENOS \\


\textbf{Asignaciones:} 

  varAsig $\rightarrow$ variable MULEQ valores $|$ variable DIVEQ valores $|$ variable MASEQ valores
  
  \hspace{15mm}$|$ variable MENOSEQ valores $|$ variable = valores$|$ ID . ID = valores 


variable $\rightarrow$ ID $|$ vecVal $|$ vecVal.varYVals  \\


\textbf{Operaciones binarias enteras:} 

 valoresMat $\rightarrow$ INT $|$ FLOAT $|$ funcInt $|$ atributos $|$ funcString
 
  \hspace{15mm} $|$ STRING $|$ varYVals $|$ varsOps $|$ (ternarioMat)  
  
 
eMat $\rightarrow$ eMat + p $|$ valoresMat + p $|$ eMat + valoresMat $|$ valoresMat + valoresMat
  
  \hspace{15mm} $|$ eMat - p  $|$ valoresMat - p  $|$ eMat - valoresMat $|$ valoresMat - valoresMat $|$ p
  
 p $\rightarrow$ p * exp $|$ p / exp $|$ p  $|$ valoresMat * exp $|$ valoresMat / exp $|$ valoresMat 
 
   \hspace{15mm} $|$ p * valoresMat $|$ p / valoresMat $|$ p \% valoresMat $|$ valoresMat * valoresMat
   
   \hspace{15mm} $|$ valoresMat \/ valoresMat $|$ valoresMat \% valoresMat $|$ exp
    
 exp $\rightarrow$ exp \^ iSing $|$ valoresMat \^ iSing $|$ exp \^ valoresMat
 
 \hspace{15mm} $|$ valoresMat \^ valoresMat $|$ iSing
  
iSing $\rightarrow$ - paren $|$ + paren $|$ - valoresMat $|$ + valoresMat $|$ paren
  
  
 paren $\rightarrow$ ( eMat )  $|$ ( valoresMat ) \\
  
  
\textbf{Expresiones booleanas:} 

valoresBool $\rightarrow$ BOOL $|$ funcBoo l$|$ varYVals $|$ varsOps $|$ (  ternarioBool )  
 
 expBool $\rightarrow$ expBool OR and $|$ valoresBool OR and $|$
 
 \hspace{15mm}$|$ expBool OR valoresBool $|$ valoresBool OR valoresBool $|$ and
  
 and $\rightarrow$ and AND eq $|$ valoresBool AND eq $|$ and AND valoresBool 
 
 \hspace{15mm}$|$ valoresBool AND valoresBool $|$ eq

 eq $\rightarrow$ eq EQEQ mayor $|$ eq DISTINTO mayor $|$ tCompareEQ EQEQ mayor 
 
 \hspace{15mm} $|$ tCompareEQ DISTINTO mayor $|$ eq EQEQ tCompareEQ  
 
 \hspace{15mm} $|$ eq DISTINTO tCompareEQ  tCompareEQ EQEQ tCompareEQ 
 
 \hspace{15mm} $|$ tCompareEQ DISTINTO tCompareEQ $|$ mayor
  
 
 tCompareEQ $\rightarrow$ BOOL $|$ funcBool $|$ varYVals $|$ varsOps $|$ INT
  
 \hspace{15mm} $|$ FLOAT $|$ funcInt $|$ eMat $|$ ( ternarioBool )  $|$ ( ternarioMat )
  
  
  tCompare $\rightarrow$ eMat $|$ varsOps $|$ varYVals $|$ INT $|$ funcInt  $|$ FLOAT $|$ ( ternarioMat ) 
  
  mayor  $\rightarrow$ tCompare $>$ tCompare $|$ menor
  
  menor  $\rightarrow$ tCompare $<$ tCompare $|$ not
  
   not  $\rightarrow$  NOT not $|$ NOT valoresBool $|$ parenBool
  
  parenBool  $\rightarrow$ ( expBool ) 
  
  
