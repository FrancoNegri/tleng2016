\subsection{Lexer}
El lexer se encarga de tomar el imput y tokenizarlo. Comenzamos el proceso de implementación del lexer declarando las palabras reservadas de nuestro lenguaje, esto es:
\\
Begin,end,while,for,if,else,do,res,return,AND,OR,NOT,print,multiplicacionEscalar,capitalizar,colineales,print,length
\\
Luego procedimos a tokenizar cosas mas complejas como strings, comentarios, numeros de punto flotante que ya requerían de la utilizacion de expresiones regulares. Para ello utilizamos el formato brindado por ply para tokenizaras.
\\
Por ejemplo, para tokenizar un string utilzamos la expresion regular:  $$\" \.\*\? \"$$

\\
Que nos dará tokenizará todas las cadenas de caracteres que comiencen y terminen con una comilla como un string
\\
Para tokenizar un comentario utilizaremos la expresión regular $$\\\# \.\*$$ que nos dará todos los caracteres que comiencen con un numeral hasta el salto de linea.