En este trabajo práctico se nos propuso realizar ingeniería de requerimientos
para desarrollar un sistema de compras online que le permita a la cadena
\textbf{Mes\%} aumentar sus ventas. Gracias a esto aprendimos la importancia de
la ingeniería de requerimientos: cómo, conociendo los requerimientos, tantos
funcionales como no-funcionales, se puede medir el alcance y viabilidad de los
sistemas planteados.

El diagrama de contexto nos ayudó a separar los requerimientos del sistema de
los objetivos en el mundo, además de ponernos en claro el comportamiento entre
agentes del modelo, para así tener una idea clara del mundo y poder detectar
los comportamientos del mundo que podrían impactar en el sistema.

El diagrama de objetivos nos ayudó a pensar estrategias para cumplir un
objetivo y compararlas contra objetivos blandos para así decidir sobre ellas.
También nos ayudó a asignar responsabilidades.

Lo primero que llevamos a cabo fue el diagrama de contexto. Esto nos ayudó a
entender y formalizar el modo en que los agentes interactúan. Luego, comenzamos
el diagrama de objetivos. Este diagrama fue el que más difícil nos resultó, ya
que fue complicado saber hasta que punto refinar los objetivos o si, al
realizar un Y-refinamiento, la descomposición obtenida es minimal y suficiente
para que el objetivo principal sea satisfecho. Luego de haber desarrollado en
profundidad el diagrama de objetivos, pudimos aclarar varias dudas que teníamos
pendientes sobre el diagrama de contexto.

Finalmente, se destaca como lo más complicado de todo el trabajo práctico --y
por mucho-- lograr un orden para imprimir el diagrama de objetivos de una forma
“decente”.
