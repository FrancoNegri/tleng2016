\subsubsection*{Página web para realizar pedidos vs Cajas de autoservicio}

Utilizar cajas de autoservicio implica contratar a una empresa para que las diseñe e instale en cada local. Ésto potencialmente toma más tiempo que el diseñar una página web. Por otra parte, al utilizar las cajas, las compras realizadas por cada cliente no son almacenadas en el sistema web. Sin embargo, los pedidos de reposición de los locales sí pasan por la página web e, indirectamente, aportan información sobre qué se vendió y qué no en cada local (aunque no con tanta granularidad).

\subsubsection*{Penalizar clientes vs confiar en el cliente vs avisar pedido en camino}

Penalizar a los clientes implica generar inconvenientes a éstos como consecuencia de no haber estado presente al momento de la entrega de un pedido previamente coordinado. Ésto podría impactar en la conformidad de los clientes.  En cambio, si no se toman medidas y la base de clientes se acostumbra a no respetar las fechas pactadas, esto generará un incremento innecesario en el costo de proveer el servicio de envíos. Esta situacion se podria mitigar enviando un SMS o realizando una llamada al cliente al momento que sale el pedido hacia la casa del cliente. Ésto requiere el esfuerzo del encargado de entregas, además de que el cliente debería proveer un número de celular o teléfono mediante el cual pueda ser contactado. También se debe tener en cuenta que el envío de SMS y los llamados telefónicos implicarán un costo extra en el envío de cada pedido.
