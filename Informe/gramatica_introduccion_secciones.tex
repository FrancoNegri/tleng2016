\documentclass[10pt, a4paper]{article}
\usepackage[english]{babel} % espanol
\usepackage{enumerate} % enumerados
\usepackage{longtable}

\input{config/paquetes.tex}
\input{config/variables.tex}

\begin{document}
\input{config/caratula.tex}	

\tableofcontents
\pagebreak

\section{introducción}
El problema planteado implica el desarrollo de un analizador léxico y sintáctico para un lenguaje de programación dado. Para ello, se buscará una gramatica que genere dicho lenguaje y además cumpla con los requisitos necesarios para construir un parser a partir de la misma. Finalmente, además de poder decidir si una cadena de texto pertenece al lenguaje, se formateará la cadena de texto de entrada para cumplir con la indentación adecuada en el caso de que la misma sea válida.

\section{Descripcion del lexer:}


\section{Descripción de la gramática}

A continuación se define la gramática utilizada para construir el parser. La misma es \textbf{no ambigua} y \textbf{LALR}. La garantía de esto es que ply acepta gramaticas de este tipo, nuestra implementacion no arroja conflictos \textbf{shift/reduce} o \textbf{reduce/reduce} \\

\linespread{1.125}\selectfont

$G$|$<V_{t}$, $V_{nt}$, $g$, $P>$

$V_{t}$ es el conjunto de símbolos terminales dado por los símbolos en \textbf{mayuscúlas} que aparecen 

en las producciones.


$V_{nt}$ es el conjunto de simbolos no-terminales dado los literales(operandores) y los símbolos

en \textbf{minúsculas} que aparecen en las producciones.



$P$ es el conjunto de producciones dadas a continuación$|$\\


\textbf{Sentencias y estructura general} \\

g $\rightarrow$ linea g $|$ COMMENT g $|$ empty 

linea $\rightarrow$ lAbierta $|$ lCerrada \\


\textit{Linea Abierta: Hay por lo menos un IF que no matchea con un else} 

lAbierta $\rightarrow$ IF (cosaBooleana) linea $|$

\hspace{15mm}$|$ IF (cosasBooleana) {g} ELSE lAbierta

\hspace{15mm}$|$ IF (cosasBooleana) {g} ELSE lAbierta

\hspace{15mm}$|$ IF (cosasBooleana) {g} 

\hspace{15mm}$|$ loop Labierta \\

\textit{Las siguientes son las variantes de tener bloques cerrados else bloques cerrados.}

\textit{Un bloque cerrado puede ser una sentencia unica o un bloque entre llaves.}

\textit{En cada uno de estos casos puede haber, o no, comentarios. De ahi todas estas combinaciones.} 


lCerrada $\rightarrow$ sentencia

\hspace{15mm}$|$ COMMENT com $|$ lambda

\hspace{15mm}$|$ IF (cosaBooleana) {g} ELSE {g}

\hspace{15mm}$|$ IF ( cosaBooleana ) lCerrada ELSE { g } 

\hspace{15mm}$|$ IF ( cosaBooleana ) COMMENT com lCerrada ELSE { g } 

\hspace{15mm}$|$ IF ( cosaBooleana ) { g } ELSE lCerrada 

\hspace{15mm}$|$ IF ( cosaBooleana ) lCerrada ELSE lCerrada 

\hspace{15mm}$|$ IF ( cosaBooleana ) COMMENT com lCerrada ELSE lCerrada 

\hspace{15mm}$|$ IF ( cosaBooleana ) lCerrada ELSE  COMMENT com lCerrada 

\hspace{15mm}$|$ IF ( cosaBooleana ) COMMENT com lCerrada ELSE  COMMENT com lCerrada

\hspace{15mm}$|$ loop { g } 

\hspace{15mm}$|$ loop lCerrada 

\hspace{15mm}$|$ loop COMMENT com lCerrada

\hspace{15mm}$|$ DO { g } WHILE ( valores ) ;
  
\hspace{15mm}$|$ DO lCerrada WHILE ( valores ) ;  
  
\hspace{15mm}$|$  DO COMMENT com lCerrada WHILE ( valores ) ; \\


\textbf{Sentencias básicas:}

sentencia $\rightarrow$ varsOps $|$ func ; $|$ varAsig ; $|$ RETURN; $|$ ; \\


\textbf{Bucles:}

loop $\rightarrow$ WHILE (valores) $|$ FOR (primerParam ; valores ; tercerParam)

primerParam $\rightarrow$  varAsig $|$ lambda

tercarParam $\rightarrow$  varsOps $|$ varAsig $|$ func $|$ lambda

cosaBooleana $\rightarrow$ expBool $|$ valoresBool \\
  

\textbf{Funciones:}

func $\rightarrow$ FuncReturn $|$ FuncVoid

funcReturn $\rightarrow$ FuncInt $|$ FuncString $|$ FuncBool

funcInt $\rightarrow$ MULTIESCALAR( valores, valores param)

funcInt $\rightarrow$ LENGTH( valores)

funcString $\rightarrow$ CAPIALIZAR(valores)

FuncBool $\rightarrow$ colineales(valores,valores )

FuncVoid $\rightarrow$ print(Valores) 

param $\rightarrow$ valores $|$ lambda \\

\textbf{Vectores y variables:}

vec $\rightarrow$ [elem]

elem $\rightarrow$ valores,elem $|$ valores

vecval $\rightarrow$ id [expresion] $|$ vec [expresion] $|$ vecVal [expresion] $|$ ID[INT]

expresion $\rightarrow$ eMat
$|$  expBool
$|$  funcReturn
$|$  reg
$|$  FLOAT $|$
$|$STRING $|$
$|$  RES

\hspace{15mm}$|$   BOOL $|$  varYVals $|$  varsOps $|$  vec $|$  atributos $|$  ternario 
 
valores $\rightarrow$  varYVals
$|$  varsOps
$|$  eMat
$|$  expBool
$|$  funcReturn
$|$  reg
$|$  INT
$|$  FLOAT
  
  \hspace{15mm}$|$  STRING
  $|$  BOOL
  $|$  ternario
  $|$  atributos
  $|$  vec
  $|$  RES
 
atributos $\rightarrow$ ID.valoresCampos $|$ reg.valoresCampos

valoresCampos $\rightarrow$ varYVals $|$ END $|$ BEGIN

\textbf{Operadores ternarios:}

ternario $\rightarrow$ ternarioMat $|$ ternarioBool $|$ (ternarioBool) $|$ (ternarioMat) 

\hspace{15mm}$|$ ternarioVars  $|$ (ternarioVars)

ternarioVars $\rightarrow$ valoresBool ? valoresTernarioVars : valoresTernarioVars  

  \hspace{15mm}$|$  valoresBool ? valoresTernarioVars : valoresTernarioMat 
  
  \hspace{15mm}$|$  valoresBool ? valoresTernarioMat : valoresTernarioVars
  
  \hspace{15mm}$|$  valoresBool ? valoresTernarioVars : valoresTernarioBool 
  
  \hspace{15mm}$|$  valoresBool ? valoresTernarioBool : valoresTernarioVars
  
  \hspace{15mm}$|$  expBool ? valoresTernarioVars : valoresTernarioVars
  
  \hspace{15mm}$|$  expBool ? valoresTernarioVars : valoresTernarioMat 
  
  \hspace{15mm}$|$  expBool ? valoresTernarioMat : valoresTernarioVars
  
  \hspace{15mm}$|$  expBool ? valoresTernarioVars : valoresTernarioBool 
  
  \hspace{15mm}$|$  expBool ? valoresTernarioBool : valoresTernarioVars  
  

  
valoresTernaioVars $\rightarrow$ reg $|$ vec $|$ ternarioVars $|$ (ternarioVars) $|$ atributos 

\hspace{15mm}$|$ varsOps  $|$ varYVals $|$ RES

valoresTernarioMat $\rightarrow$ valoresBool ? valoresTernarioMat : valoresTernarioMat  

\hspace{15mm}$|$ expBool ? valoresTernarioMat : valoresTernarioMat
  
valoresTernarioMat $\rightarrow$ INT $|$ FLOAT $|$ funcInt $|$ STRING $|$ eMAt 

\hspace{15mm}$|$ ternarioMat $|$ (ternarioMat)

ternarioBool $\rightarrow$ valoresBool ? valoresTernarioBool : valoresTernarioBool  

\hspace{15mm}$|$ expBool ? valoresTernarioBool : valoresTernarioBool
  
valoresTernarioBool $\rightarrow$ BOOL $|$ funcBool $|$ ternarioBool $|$ ( ternarioBool )$|$ expBool \\
   

\textbf{varYVals:}

varYVals $\rightarrow$ ID $|$ vecVal $|$ vecVal.varYVals \\


\textbf{Registros:} 

reg $\rightarrow$ {campos}

campos $\rightarrow$ ID:valores, campos $|$ ID:valores \\


\textbf{Operadores de variables:}

varsOps$\rightarrow$ MENOSMENOS varYVals $|$ MASMAS varYVals

\hspace{15mm}$|$ varYVals MASMAS $|$ varYVals MENOSMENOS 


\textbf{Asignaciones:} 

  varAsig $\rightarrow$ variable MULEQ valores $|$ variable DIVEQ valores $|$ variable MASEQ valores
  
  \hspace{15mm}$|$ variable MENOSEQ valores $|$ variable = valores$|$ ID . ID = valores 


variable $\rightarrow$ ID $|$ vecVal $|$ vecVal.varYVals 


\textbf{Operaciones binarias enteras:} 

 valoresMat $\rightarrow$ INT $|$ FLOAT $|$ funcInt $|$ atributos $|$ funcString
 
  \hspace{15mm} $|$ STRING $|$ varYVals $|$ varsOps $|$ (ternarioMat)  
  
 
eMat $\rightarrow$ eMat + p $|$ valoresMat + p $|$ eMat + valoresMat $|$ valoresMat + valoresMat
  
  \hspace{15mm} $|$ eMat - p  $|$ valoresMat - p  $|$ eMat - valoresMat $|$ valoresMat - valoresMat $|$ p
  
 p $\rightarrow$ p * exp $|$ p / exp $|$ p  $|$ valoresMat * exp $|$ valoresMat / exp $|$ valoresMat 
 
   \hspace{15mm} $|$ p * valoresMat $|$ p / valoresMat $|$ p \% valoresMat $|$ valoresMat * valoresMat
   
   \hspace{15mm} $|$ valoresMat \/ valoresMat $|$ valoresMat \% valoresMat $|$ exp
    
 exp $\rightarrow$ exp \^ iSing $|$ valoresMat \^ iSing $|$ exp \^ valoresMat
 
 \hspace{15mm} $|$ valoresMat \^ valoresMat $|$ iSing
  
iSing $\rightarrow$ - paren $|$ + paren $|$ - valoresMat $|$ + valoresMat $|$ paren
  
  
 paren $\rightarrow$ ( eMat )  $|$ ( valoresMat ) \\
  
  
\textbf{Expresiones booleanas:} 

valoresBool $\rightarrow$ BOOL $|$ funcBoo l$|$ varYVals $|$ varsOps $|$ (  ternarioBool )  
 
 expBool $\rightarrow$ expBool OR and $|$ valoresBool OR and $|$
 
 \hspace{15mm}$|$ expBool OR valoresBool $|$ valoresBool OR valoresBool $|$ and
  
 and $\rightarrow$ and AND eq $|$ valoresBool AND eq $|$ and AND valoresBool 
 
 \hspace{15mm}$|$ valoresBool AND valoresBool $|$ eq

 eq $\rightarrow$ eq EQEQ mayor $|$ eq DISTINTO mayor $|$ tCompareEQ EQEQ mayor 
 
 \hspace{15mm} $|$ tCompareEQ DISTINTO mayor $|$ eq EQEQ tCompareEQ  
 
 \hspace{15mm} $|$ eq DISTINTO tCompareEQ  tCompareEQ EQEQ tCompareEQ 
 
 \hspace{15mm} $|$ tCompareEQ DISTINTO tCompareEQ $|$ mayor
  
 
 tCompareEQ $\rightarrow$ BOOL $|$ funcBool $|$ varYVals $|$ varsOps $|$ INT
  
 \hspace{15mm} $|$ FLOAT $|$ funcInt $|$ eMat $|$ ( ternarioBool )  $|$ ( ternarioMat )
  
  
  tCompare $\rightarrow$ eMat $|$ varsOps $|$ varYVals $|$ INT $|$ funcInt  $|$ FLOAT $|$ ( ternarioMat ) 
  
  mayor  $\rightarrow$ tCompare $>$ tCompare $|$ menor
  
  menor  $\rightarrow$ tCompare $<$ tCompare $|$ not
  
   not  $\rightarrow$  NOT not $|$ NOT valoresBool $|$ parenBool
  
  parenBool  $\rightarrow$ ( expBool ) 
  
  
  
  
  
  

\section{Descripción de la implementación}

\section{Código}

\section{Conclusiones}

\end{document}

