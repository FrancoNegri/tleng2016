\subsection{Implementación Del Lexer}

Para la construcción del lexer se definió un conjunto de literales con operadores y otros símbolos propios del lenguaje.

\textbf{literals} = $[+,- , * , / , ^ ,\% , < , > , = , ! , \{\ , \}\ , ( , ) , [ , ] , ? , \textbf{:} , \textbf{;} , \textbf{,} , \textbf{.} ]$

A su vez, utilizamos otra funcionalidad de ply para definir las palabras reservadas del lenguaje: \\

\textbf{reserved} = $\{$
'begin' : 'BEGIN', 'end' : 'END', 'while' : 'WHILE', 'for' : 'FOR',

'if' : 'IF', 'else' : 'ELSE','do' : 'DO', 'res' : 'RES', 'return' : 'RETURN',

'AND' : 'AND', 'OR' : 'OR', 'NOT' : 'NOT', 'print' : 'PRINT',

'multiplicacionEscalar': 'MULTIESCALAR', 'capitalizar': 'CAPITALIZAR', 

'colineales': 'COLINEALES', 'print': 'PRINT', 'length': 'LENGTH',
$\}$ \\

Esto permitió evitar tener que definir demasiadas reglas simples para este tipo de operadores o palabras claves.

Para el resto de los tokens definidos, fue necesario utilizar expresiones regulares, se muestra mas abajo como se definieron cada una de ellos: (Por claridad, se omite el resto del código para la regla de los tipos,ya que  es análoga a la de \textbf{string}):

\begin{multicols}{2}

t\_EQEQ = r"=="

t\_DISTINTO = r"!="

t\_MENOSEQ = r"-="

t\_MASEQ = r"$\setminus$+="

t\_MULEQ = r"$\setminus$*="

t\_DIVEQ = r"/="

t\_MASMAS = r"$\setminus$+$\setminus$+"

t\_MENOSMENOS = r"--"

\columnbreak

def t\_STRING(token):

\hspace{5mm}    r' ' ' \" .*? \" ' ' '
    
\hspace{5mm}    atributos = $\{ \} $
    
\hspace{5mm}    atributos["type"] = "string"
    
\hspace{5mm}    atributos["value"] = token.value
    
\hspace{5mm}    token.value = atributos
    
\hspace{5mm}    return token
 
\end{multicols} 
    
   
\begin{multicols}{2}   

def t\_BOOL(token) : 

\hspace{5mm}    r"true $|$ false"
    
    
def t\_FLOAT(token):

\hspace{5mm}    r"[-]?[0-9] 
    
def t\_INT(token) : 

\hspace{5mm}    r"[-]?[1-9][0-9]* | 0"
    
\columnbreak

def t\_ID(token):

\hspace{5mm}r"[a-zA-Z\_][a-zA-Z\_0-9]*"

\hspace{5mm}t = token.value
    
\hspace{5mm}tipo = reserved.get(token.value) 

\hspace{5mm}if t.lower() not in reserved and t.upper() 

\hspace{5mm}not in reserved and t != 

\hspace{5mm}"multiplicacionEscalar":

\hspace{10mm}tipo = 'ID'

\hspace{5mm}token.type = tipo
  
\end{multicols}

\begin{multicols}{2}

def t\_NEWLINE(token):

\hspace{5mm}  r"$\setminus$n+"
  
\hspace{5mm}  token.lexer.lineno += len(token.value)


def t\_error(token):

 \hspace{2mm}message = "Token desconocido:"
    
\hspace{2mm}message += "$\setminus$n type:" + token.type
    
\hspace{2mm}message += "$\setminus$n value:" + str(token.value)
    
\hspace{2mm}message += "$\setminus$n line:" + str(token.lineno)
    
\hspace{2mm}message += "$\setminus$n position:"+str(token.lexpos)
    
\hspace{2mm}raise Exception(message)
  

\columnbreak

def t\_COMMENT(token):

\hspace{2mm}    r'$\#.*$'

t\_ignore  = ' $\setminus$ t'

\end{multicols}

\subsection{Implementacion Del Parser}

La implementación del parser consistió en transcribir la gramática final del apartado $2$ a la sintaxis de ply.

De esta manera, dada una producción:

$$Valores \rightarrow ExpresionMatematica$$

Fue necesario reescribirla como:


$def\; p\_valores(subexpressions):$

$\quad'''\; valores\; :\; ExpresionMatematica\;'''$


Ademas contamos con la funcionalidad de ply que, una vez que se a utilizado una producción permite ejecutar código adicional. Siguiendo el ejemplo anterior, suponiendo que cada vez que el parser utiliza la producción antedicha deseo imprimirla, puedo escribir:
\\

$def\; p\_valores(subexpressions):$

$\quad'''\; valores\; :\; ExpresionMatematica\;'''$

$\quad print\; subexpressions[1]$


Esto lo utilizamos tanto para escribir el output formateado con la salida correcta como para realizar el chequeo de tipos.

\subsubsection{Implementación Salida}

Para la implementación de la salida cada producción de la gramática tendrá un atributo sintetizado "value" en el que se guardará el texto de salida basándose en el valor de cada uno de los terminales y no terminales que lo componen. Esto es relativamente sencillo para las sentencias que solo requieren escribir sus terminales y no terminales seguidos de un salto de linea, pero requerirá un cuidado especial para el caso de los condicionales y los loops, ya que estos necesitan saltos de linea en lugares intermedios y una indentación adecuada.
\\
Para ejemplificar que debemos hacer utilizamos la producción:

$$lAbierta \rightarrow IF\; (\; cosaBooleana\; )\; COMMENT\; com\; lCerrada\; ELSE\; \{\; g\; \}$$ 

como caso de estudio.
\\
Lo que querremos guardar en el atributo value de lAbierta es primero un $IF\; (\; cosaBooleana\; )\;$. Nótese aquí que cosaBooleana es un conjunto de símbolos que que deben reducir como ultima instancia a una Expresión Booleana. Queda a cargo de cosaBooleana la responsabilidad de saber como imprimir cada uno de esos posibles términos que se encuentren presentes.

Esto estará seguido de un salto de linea, seguido de un comentario (con un tab) y un salto de linea, luego un no terminal que contendrá cero o uno o varios comentarios (todos conteniendo su indentación apropiada), otro salto de linea con una palabra reservada else, unas llaves que abren y una o varias sentencias indentadas un corchete que cierra.
\\
En primera instancia, la salida que nos dará este if es la deseada, sin embargo puede que ocurra el caso donde tenemos dos ifs anidados. En este caso lo esperable es que el segundo if (el interno) este indentado y que las sentencias dentro de ese if tengan dos tabs en vez de uno. Luego, no es suficiente con agregar tabs en los lugares adecuados, también necesitamos señalizar donde comienza una nueva linea y en el caso de que sea necesario agregar varios tabs, que se pueda recorrer la salida agregándolos donde se necesite.
\\
Para el manejo de la indentación utilizaremos el comienzo de cada nueva linea.

\subsubsection{Implementación Del Chequeo de Tipos}

\subsubsection{Resumen atributos}

Para el chequeo de tipos utilizamos atributos sintetizados que se enumeran a continuación:

\begin{itemize}
\item var : Denota la variable de la expresión (si es que hay solo una variable).
\item type : Denota el tipo de la expresión.
\item typeVec: Si la expresión es de tipo vector denota el tipo de sus elementos.
\item campo: denota el valor de un campo de un registro.
\item campos: Denota los campos en un registro en forma de diccionario donde la clave es el nombre del campo y el valor es el tipo del campo.
\item varAsig: Denota la variable en asignaciones como a[2] = 1, solo para chequear que el tipo de la variable es igual al valor asignado.
\item esVector: Se utiliza en asignaciones para saber si la variable a asignar es de tipo vector (se explica mas adelante).
\end{itemize}

\subsubsection{Asignación con variables, vectores y registros}
Los tipos posibles en el atributo \textit{type} son: string, float, int, bool, vec y reg. En algunas producciones se usan como simples atributos sintetizados, en el caso de que una expresión sea una variable se necesita saber que tipo tuvo esa variable cuando se inicializó. 

Para esto se cuenta con el diccionario global \textit{variables} que tiene de claves a las variables y de valor su tipo (con fines declarativos hay otro diccionario dentro del valor, que tiene una clave "type" y el valor es el tipo de la variable).

Algo similar ocurre con variables que referencian a  vectores. Se agrega la variable del vector a \textit{variables} con un atributo adicional \textit{typeVec} que denota el tipo del vector.

En caso de los registros se guarda un atributo \textit{campos} en donde se guardan los campos del registro. En el se guarda un diccionario, donde la clave corresponde al nombre del campo y el valor al tipo del campo.

En caso de las asignaciones en variables de tipos registro se guarda, similar a \textit{variables}, un diccionario global \textit{registros} de igual manera al atributo \textit{campos} de los registros.

\subsubsection{Para ejecución}
Cuando se establece en un atributo type o var el string "Para ejecucion" significa que el parser admite la cadena pero se tiene que chequear en tiempo de ejecución su tipo. 

Las operaciones que se delega el chequeo de tipos a tiempo de ejecución son:

\begin{itemize}
\item Chequeo de que un indice en un acceso a vector este en rango.
\item Chequeo de tipos en variables dentro de un vector o un registro (por ejemplo a = [b]; \\ a[0][1] = 1; c = \{vector: b\}; c.vector++;, donde b es de tipo vector).

\end{itemize}

\subsubsection{Valores de vector y asignaciones}
VecVal es la regla que permite crear accesos a vectores (por ejemplo a[5]). Esta regla no realiza chequeo de indice. Establece el tipo de retorno al acceso del vector.

Consideremos el siguiente ejemplo:

$ a[1] = [1,2,3];$
$ c = b[1] $

Tanto $ a[1]$ como $ b[1]$ provienen de la producción \textbf{vecVal}. El problema es que estas dos expresiones tienen distinto significado. $ b[1]$ corresponde a un valor de un vector, no interesa la variable del vector. $ a[1]$, en cambio, refiere a la posición del vector y por lo tanto se requiere saber la variable de dicho vector para poder actualizar su tipo.

Para solucionar esto creamos un no terminal nuevo llamado \textbf{Variable} el cual se encarga de establecer el nombre de la variable del vector y ademas guarda un booleano \textit{esVector} para saber si dicha expresión es un vector. También en VecVal guardamos la información de la variable en el atributos \textit{varAsig}.

Si la expresión de la izquierda de la asignación no tiene un atributo "var" entonces tira una excepción "Solo se puede acceder a variables.

Ademas en operaciones como \textit{(a = 2) ? 5 : 2}, se establece como tipo de la asignación el tipo de la derecha (en este caso es el tipo del número 2 que es \textit{int}) y como variable el atributo var de la expresión de la izquierda (en este caso \textit{a}).

